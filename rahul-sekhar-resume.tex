%% start of file `template.tex'.
%% Copyright 2006-2013 Xavier Danaux (xdanaux@gmail.com).
%
% This work may be distributed and/or modified under the
% conditions of the LaTeX Project Public License version 1.3c,
% available at http://www.latex-project.org/lppl/.


\documentclass[11pt,a4paper,sans]{moderncv}        % possible options include font size ('10pt', '11pt' and '12pt'), paper size ('a4paper', 'letterpaper', 'a5paper', 'legalpaper', 'executivepaper' and 'landscape') and font family ('sans' and 'roman')

% modern themes
\moderncvstyle{classic}                            % style options are 'casual' (default), 'classic', 'oldstyle' and 'banking'
\moderncvcolor{blue}                                % color options 'blue' (default), 'orange', 'green', 'red', 'purple', 'grey' and 'black'
%\renewcommand{\familydefault}{\sfdefault}         % to set the default font; use '\sfdefault' for the default sans serif font, '\rmdefault' for the default roman one, or any tex font name
%\nopagenumbers{}                                  % uncomment to suppress automatic page numbering for CVs longer than one page

% character encoding
\usepackage[utf8]{inputenc}                       % if you are not using xelatex ou lualatex, replace by the encoding you are using
%\usepackage{CJKutf8}                              % if you need to use CJK to typeset your resume in Chinese, Japanese or Korean

% adjust the page margins
\usepackage[scale=0.75]{geometry}
%\setlength{\hintscolumnwidth}{3cm}                % if you want to change the width of the column with the dates
%\setlength{\makecvtitlenamewidth}{10cm}           % for the 'classic' style, if you want to force the width allocated to your name and avoid line breaks. be careful though, the length is normally calculated to avoid any overlap with your personal info; use this at your own typographical risks...

\usepackage{import}

\renewcommand{\labelitemi}{\scriptsize\color{black}{$\bullet$}}

% \usepackage{hyperref}
\definecolor{linkcolour}{rgb}{0.22,0.45,0.7}     % hyperlinks setup
\definecolor{light}{rgb}{0.5,0.5,0.5}     % hyperlinks setup
% \AfterPreamble{\hypersetup{colorlinks,breaklinks,urlcolor=red, linkcolor=red}}

% personal data
\name{Rahul}{Sekhar}
% \title{Curriculum Vitae}                               % optional, remove / comment the line if not wanted
\address{93, 30\textsuperscript{th} cross road}{Banashankari Stage 2}{Bangalore -- 560070, India}% optional, remove / comment the line if not wanted; the "postcode city" and and "country" arguments can be omitted or provided empty
\phone[mobile]{+91 9916746045}                   % optional, remove / comment the line if not wanted
% \phone[fixed]{01234 123456}                    % optional, remove / comment the line if not wanted
%\phone[fax]{+3~(456)~789~012}                      % optional, remove / comment the line if not wanted
\email{sekhar.rahul@gmail.com}                               % optional, remove / comment the line if not wanted
% \homepage{www.myname.webs.com}                         % optional, remove / comment the line if not wanted
%\extrainfo{additional information}                 % optional, remove / comment the line if not wanted
% \photo[64pt][0.4pt]{pic}                       % optional, remove / comment the line if not wanted; '64pt' is the height the picture must be resized to, 0.4pt is the thickness of the frame around it (put it to 0pt for no frame) and 'picture' is the name of the picture file
%\quote{Some quote}                                 % optional, remove / comment the line if not wanted

% to show numerical labels in the bibliography (default is to show no labels); only useful if you make citations in your resume
%\makeatletter
%\renewcommand*{\bibliographyitemlabel}{\@biblabel{\arabic{enumiv}}}
%\makeatother
%\renewcommand*{\bibliographyitemlabel}{[\arabic{enumiv}]}% CONSIDER REPLACING THE ABOVE BY THIS

% bibliography with mutiple entries
%\usepackage{multibib}
%\newcites{book,misc}{{Books},{Others}}
%----------------------------------------------------------------------------------
%            content
%----------------------------------------------------------------------------------
\begin{document}
%\begin{CJK*}{UTF8}{gbsn}                          % to typeset your resume in Chinese using CJK
%-----       resume       ---------------------------------------------------------
\makecvtitle

\small{I find great satisfaction in seeing problems tackled completely and elegantly. I involve myself wholeheartedly in everything I do and pride myself on my ability to pick up concepts and techniques quickly.}

\vspace{6pt}

\section{Skills}

\vspace{6pt}

\cvline
{\textcolor{light}{Languages}}
{In-depth knowledge of Javascript, Ruby, HTML and CSS. Experience with SQL, Python, PHP and shell scripting.}

\vspace{6pt}

\cvline
{\textcolor{light}{Frameworks}}
{In-depth experience with AngularJS, Rails and Wordpress. Experience with EmberJS, Backbone, Bootstrap, requireJS and AMD modules.}

\vspace{6pt}

\cvline
{\textcolor{light}{Methodology}}
{A strong focus on readable, modular, maintainable code. Experience with agile development. Much of the architecture I've designed has been based on the 12 factor pattern.}

\vspace{6pt}

\cvline
{\textcolor{light}{Web}}
{I make it a point to follow best practices and keep track of new developments and techniques. I have designed REST APIs, optimised webapps, worked with CDNs and mobile-friendly responsive layouts.}


\vspace{6pt}

\cvline
{\textcolor{light}{Tools}}
{Source control with git. Build processes with grunt and gulp. SASS, LESS and Compass as precompilers. Automated deployment with capistrano and githooks.}

\vspace{6pt}


\cvline
{\textcolor{light}{Testing}}
{BDD, TDD and thorough unit and integration testing using Jasmine, RSpec, Cucumber and Selenium. I believe tested code is a necessity not a luxury.}

\cvline
{\textcolor{light}{Personal}}
{I have strong communication skills and am able to convey my ideas clearly and straightforwardly without jargon. I have lead a team and mentored developers, to produce quality code.}

\vspace{6pt}

\vspace{6pt}

\section{Work Experience}

\vspace{6pt}

\cventry
{\textcolor{light}{2015--current}}
{Member of Technical Staff}
{}
{Nutanix}{Bangalore}
{I lead a team of front end developers at Nutanix, working on the interface for a number of technical tools, including sizing and quote generation for clusters, automated deployment and migration of workloads and healthcheck tools for clusters and migrated workloads. The company sets a great deal of importance on design and UX, and we ensure our interfaces are intuitive, clean, well designed and robust. My work here involves:
\vspace{6pt}
\begin{itemize}
  \item Front end architectural decisions based on the requirements of each tool.
  \item Using modern tools and workflows to ensure code is well written and tested yet flexible and quick to write.
  \item Agile, scrum based methodology with two-week sprints for each project.
  \item Monitoring code quality across simultaneous projects.
  \item Interfacing with designers, stakeholders and technical experts to produce an intuitive, useful and well designed product.
  \item Refactoring and improving the workflow for legacy projects.
\end{itemize}}

\vspace{8pt}

\cventry
{\textcolor{light}{2012--2015}}
{Web Architect}
{}
{Kairi Studios}{Bangalore}
{At Kairi Studios I worked on webapps and sites for our clients from end to end -- technical design, development and deployment. While I worked with the full stack here, due to the nature of our projects my focus was on front-end development. We received a great deal of appreciation for our interfaces, and the maintainability and robustness of our output.
\endgraf
\vspace{6pt}
Some of our projects included a redesign and extensive administration interface for NCF India, an internal academics and journaling system for small schools, and and an online store for childrens books. My work at Kairi involved:
\vspace{6pt}
\begin{itemize}
  \item Architecture and tooling decisions based on client requirements.
  \item Behaviour driven development, with user focused stories and thorough automated testing.
  \item Projects that used AngularJS, Backbone, Ruby on Rails and Wordpress.
  \item Responsive, mobile-friendly layouts.
  \item Modern, creative and best practice based development.
  \item Documentation and maintenance of projects.
\end{itemize}}

\vspace{8pt}

\cventry
{\textcolor{light}{2011--2012}}
{Web Developer}
{}
{Freelance}{}
{I put together websites in collaboration with some very talented designers as a freelance developer. My focus was to find creative solutions to problems and write clean, maintainable code that followed best practices. I made it a point to explore new technologies, frameworks, tools and techniques in an effort to stay up-to-date and remain flexible in my infrastructure choices, which has served me well.}

\vspace{6pt}


\section{Education}

\vspace{6pt}

\cventry
{\textcolor{light}{2007--2011}}
{BMS College of Engineering}
{Bangalore}
{}
{}
{B.E. in Telecommunication Engineering -- First Class}

\vspace{6pt}

\cventry
{\textcolor{light}{1995--2007}}
{The Valley School}
{Bangalore}
{}
{}
{Indian School Certificate (ISC) -- First class}

\vspace{6pt}

\section{Interests and extra-curricular activity}

\vspace{6pt}

\cvline
{\textcolor{light}{Music}}
{A passionate guitarist, I play in a small band and am in the process of writing and recording material for our first album. I have also been avidly involved in Tango dancing.}

\vspace{6pt}

\cvline
{\textcolor{light}{Hiking}}
{I enjoy treks and have completed a basic mountaineering course at the Himalayan Mountaineering Institute, Darjeeling with an A grade.}

\vspace{6pt}

\cvline
{\textcolor{light}{Sport}}
{A cyclist and runner, I've taken part in cycle tours and numerous runs.}

\vspace{6pt}

% Publications from a BibTeX file without multibib
%  for numerical labels: \renewcommand{\bibliographyitemlabel}{\@biblabel{\arabic{enumiv}}}% CONSIDER MERGING WITH PREAMBLE PART
%  to redefine the heading string ("Publications"): \renewcommand{\refname}{Articles}
\nocite{*}
\bibliographystyle{plain}
\bibliography{publications}                        % 'publications' is the name of a BibTeX file

% Publications from a BibTeX file using the multibib package
%\section{Publications}
%\nocitebook{book1,book2}
%\bibliographystylebook{plain}
%\bibliographybook{publications}                   % 'publications' is the name of a BibTeX file
%\nocitemisc{misc1,misc2,misc3}
%\bibliographystylemisc{plain}
%\bibliographymisc{publications}                   % 'publications' is the name of a BibTeX file

%-----       letter       ---------------------------------------------------------

\end{document}


%% end of file `template.tex'.
